\section{Getting UNISIM}
\label{getting_unisim}

\subsection{Source distributions}

The source of UNISIM is composed of three different packages:

\begin{itemize}
\item unisim\_tools: utilities used by the simulators and other external tools
\item unisim\_lib: library with all the modules available in UNISIM, and the core libraries of the UNISIM framework
\item unisim\_simulators: different simulators built using modules from the unisim\_lib
\end{itemize}

\subsubsection{Tarballs}

Source is available as tarballs (.tar.gz) at \url{http://unisim.org/distributions/source}, where you will find the three different packages.
To uncompress a source tarball, use program ‘tar’ at the command prompt:

\begin{verbatim}
$ tar zxvf <tarball name>
\end{verbatim}

To compile (and install) the packages refer to the corresponding sections. Note that the compilation order is:

\begin{enumerate}
\item unisim\_tools, check UNISIM Tools page
\item unisim\_lib, check UNISIM Library page
\item unisim\_simulators, check UNISIM simulators page
\end{enumerate}

\subsubsection{Subversion repository}

Source of UNISIM is also available from the subversion repository at \url{https://guest@unisim.org/svn/devel}.
Note that the subversion repository hosts the development version of UNISIM which may have installation problems or instabilities.
You may use the subversion repository only if you want to have a daily up-to-date version of UNISIM and don’t care about such problems.
To access to the source, you must have a subversion client installed which is available for you operating system either as a package or an installer here: \url{http://tortoisesvn.tigris.org}.

To download the source from the command prompt, just type:

\begin{verbatim}
$ svn --username guest --password "" co https://unisim.org/svn/devel
\end{verbatim}

User “guest” has read permission to the source tree. No password is required for user “guest”.

\subsection{Binary distributions}

Binary distributions are only provided for UNISIM Tools and UNISIM Simulators. UNISIM Library is only distributed as source code.
You must have permission to install softwares to use these binary distributions.
On Linux or any other Unix-like operating systems, this means that you can login as user “root”, or you can run commands as root using “sudo”.
On Windows, this means that you are either administrator or a user with some privileges.

\subsubsection{Binaries for Debian, Ubuntu, ...}

\noindent \textbf{Manually installing the UNISIM debian packages}

The Debian packages (.deb) are available here: \url{http://unisim.org/distributions/binary/deb}.
These packages can be used on Debian Linux or any derived debian distributions such as Ubuntu
Program ‘dpkg’ can install the UNISIM debian packages:

\begin{verbatim}
$ dpkg -i <package name>
\end{verbatim}

Symetrically, it can uninstall the UNISIM debian packages:

\begin{verbatim}
$ dpkg -e <package name>
\end{verbatim}

\noindent \textbf{Debian Package repository}

Your Linux can be configured to fetch the packages using you preferred package manager such as apt-get, synaptic or adept.
At the command prompt, type the following as root:

\begin{verbatim}
$ echo "deb http://unisim.org/distributions/binary/deb stable main" > \
  /etc/apt/sources.list.d/unisim.list
$ apt-get update
\end{verbatim}

To install the UNISIM packages from the command prompt, type the following as root:

\begin{verbatim}
$ apt-get install unisim-tools
$ apt-get install unisim-simulators
\end{verbatim}

To remove the UNISIM packages from the command prompt, type the following as root:

\begin{verbatim}
$ apt-get remove unisim-tools
$ apt-get remove unisim-simulators
\end{verbatim}

\subsubsection{Binaries for Redhat, Fedora, Mandriva, Suse ...}

\noindent \textbf{Manually installing the UNISIM rpm packages}

The Redhat packages (.rpm) are available here: \url{http://unisim.org/distributions/binary/rpm}.
These packages can be used on Redhat Linux or any derived redhat distributions such as Mandriva or Fedora.
Program ‘rpm’ can install the UNISIM rpm packages:

\begin{verbatim}
$ rpm -i <package name>
\end{verbatim}

Symetrically, it can uninstall the UNISIM rpm packages:

\begin{verbatim}
$ rpm -e <package name>
\end{verbatim}

\noindent \textbf{Mandriva Package repository}

Your Mandriva Linux can be configured to fetch the packages using you package manager such as urpmi, or drakerpm.
At the command prompt, type the following as root:

\begin{verbatim}
$ urpmi.addmedia UNISIM http://unisim.org/distributions/binary/rpm
\end{verbatim}

To install the UNISIM packages from the command prompt, type the following as root:

\begin{verbatim}
$ urpmi unisim-tools
$ urpmi unisim-simulators
\end{verbatim}

To remove the UNISIM packages from the command prompt, type the following as root:

\begin{verbatim}
$ urpme unisim-tools
$ urpme unisim-simulators
\end{verbatim}

\subsubsection{Binaries for Windows (2000/XP/XP64/Vista 32/Vista 64)}

The Windows installers (.exe) are available here:\\
\url{http://unisim.org/distributions/binary/win32}.\\
Run the installers and follow the instructions.
To develop with UNISIM on Windows you may want to use Mingw32/MSYS, which is available for download at \url{http://www.mingw.org}.\\
You can also use a “ready to use” packaged version of Mingw32/MSYS available here:\\
\url{http://unisim.org/distributions/binary/win32/mingw32-unisim-pack.exe}.\\
To uninstall UNISIM, go to “Start” $\rightarrow$ “Control Panel” $\rightarrow$ “Add or Remove Programs”, and select the UNISIM component you want to uninstall.

\noindent \textbf{Tip:} When developing under Windows with Mingw32/MSYS, you should avoid using spaces in file and directory names

\section{Building UNISIM}

\subsection{Building the UNISIM Tools}

\subsubsection{Requirements}

\begin{itemize}
\item GNU Bash shell (3.2.25 recommended)
\item GNU Make (3.81 recommended)
\item GNU C++ Compiler (g++) ($>=$ 3.x)
\item automake ($>=$ 1.9.6 recommended)
\item autoconf ($>=$ 2.59 recommended)
\item bison (2.3 recommended) or Berkeley YACC (1.9 recommended)
\item flex (2.5.4 recommended)
\end{itemize}

\subsubsection{Building the 'configure' scripts}

If you've downloaded the source code as a tarball, you can ignore this paragraph.
If you've downloaded the source code with subversion, you first need to build the configure scripts:

\begin{verbatim}
$ cd unisim_tools
$ ./build-configure.sh
\end{verbatim}

In case of failure while running ‘build-configure.sh’, check that ‘find’, ‘autoconf’, ‘autoheader’ and ‘automake’ are correctly installed.

\subsubsection{Configuring}

\begin{verbatim}
$ cd unisim_tools
$ ./configure --prefix=${HOME}/unisim
\end{verbatim}

Script configure also supports the following options:

\begin{itemize}
\item \texttt{--disable-<directory>}: Disable compilation of a whole subdirectory\\
(e.g. \texttt{--disable-build\_tool-genisslib} prevents configure and make from entering directory genisslib)
\end{itemize}

\subsubsection{Compiling the source code}

\begin{verbatim}
$ make
\end{verbatim}

\subsubsection{Installing the UNISIM Tools}

\begin{verbatim}
$ make install
\end{verbatim}


\subsection{Building the UNISIM Library}

\subsubsection{Requirements}

\noindent \textbf{Mandatory}

\begin{itemize}
\item UNISIM Tools
\item GNU Bash shell (3.2.25 recommended)
\item GNU Make (3.81 recommended)
\item GNU C++ Compiler (g++) ($>=$ 3.x)
\item automake ($>=$ 1.9.6 recommended)
\item autoconf ($>=$ 2.59 recommended)
\item boost development library (1.34.0 recommended)
\item libxml2 development library (2.6.30 recommended)
\item zlib development library (1.2.3 recommended)
\item Core SystemC Language ($>=$ 2.1)
\item TLM Transaction Level Modeling Library, Release 2.0
\end{itemize}

\noindent \textbf{Optional}

\begin{itemize}
\item libSDL development library provides video frame capability and access to input devices (e.g. keyboard)
\item editline (libedit) development library provides line editing and history capabilities (e.g. for an inline debugger)
\item ncurses development library provides access to terminal screen (e.g. for an inline debugger)
\item Cacti 4.2 as a C++ class library (see \url{https://guest@unisim.org/svn/devel/extra/cacti}) provides a foundation to estimate power consumption
\end{itemize}

\subsubsection{Building the UNISIM Library 'configure' scripts}

If you've downloaded the UNISIM Library source code as a tarball, you can ignore this paragraph.
If you've downloaded the UNISIM Library source code with subversion, you first need to build the configure scripts:

\begin{verbatim}
$ cd unisim_lib
$ ./build-configure.sh
\end{verbatim}

In case of failure while running \texttt{build-configure.sh}, check that \texttt{find}, \texttt{autoconf}, \texttt{autoheader} and \texttt{automake} are correctly installed.

\subsubsection{Configuring UNISIM Library}

\begin{verbatim}
$ cd unisim_lib
$ ./configure --prefix=${HOME}/unisim \
              --with-unisim-tools=${HOME}/unisim \
              --with-systemc=${HOME}/systemc \
              --with-tlm20=${HOME}/TLM-2008-06-09
\end{verbatim}

Script configure also supports the following options:

\begin{itemize}
\item \texttt{--with-sdl=<path-to-sdl-install>}: Override the search path for SDL C headers
\item \texttt{--with-boost=<path-to-boost-install>}: Override the search path for boost C++ headers
\item \texttt{--with-libxml2=<path-to-libxml2-install>}: Override the search path for libxml2 C headers
\item \texttt{--with-zlib=<path-to-zlib-install>}: Override the search path for zlib C headers
\item \texttt{--with-libedit=<path-to-libedit-install>}: Override the search path for libedit C headers
\item \texttt{--with-ncurses=<path-to-ncurses-install>}: Override the search path for ncurses C headers
\item \texttt{--with-systemc=<path-to-systemc-install>}: Override the search path for SystemC C++ headers
\item \texttt{--with-tlm20=<path-to-tlm2.0-install>}: Override the search path for TLM 2.0 C++ headers
\item \texttt{--with-cacti=<path-to-cacti4.2-install>}: Override the search path for Cacti 4.2 C++ headers
\item \texttt{--enable-release}: Compile both debug and release versions of components. Debug versions are more verbose than release versions
\item \texttt{--disable-<directory>}: Disable compilation of a whole subdirectory\\ (e.g. \texttt{--disable-unisim-service-os-linux\_os} prevents configure and make from entering directory unisim/service/os/linux\_os)
\end{itemize}

\subsubsection{Compiling the UNISIM Library source code}

\begin{verbatim}
$ make
\end{verbatim}

\subsubsection{Installing the UNISIM Library}

\begin{verbatim}
$ make install
\end{verbatim}


\subsection{Building the UNISIM Simulators}

\subsubsection{Requirements}

\noindent \textbf{Mandatory}

\begin{itemize}
\item UNISIM Library
\item GNU Bash shell (3.2.25 recommended)
\item GNU Make (3.81 recommended)
\item GNU C++ Compiler (g++) ($>=$ 3.x)
\item automake ($>=$ 1.9.6 recommended)
\item autoconf ($>=$ 2.59 recommended)
\item boost development library (1.34.0 recommended)
\item libxml2 development library (2.6.30 recommended)
\item zlib development library (1.2.3 recommended)
\item Core SystemC Language ($>=$ 2.1)
\item TLM Transaction Level Modeling Library, Release 2.0
\end{itemize}

\noindent \textbf{Optional}

\begin{itemize}
\item libSDL development library provides video frame capability and access to input devices (e.g. keyboard)
\item editline (libedit) development library provides line editing and history capabilities (e.g. for an inline debugger)
\item ncurses development library provides access to terminal screen (e.g. for an inline debugger)
\item Cacti 4.2 as a C++ class library (see \url{https://guest@unisim.org/svn/devel/extra/cacti}) provides a foundation to estimate power consumption
\end{itemize}

\subsubsection{Building the 'configure' scripts}

If you downloaded the source code as a tarball, you can ignore this section. If you downloaded the source code with subversion, you first need to build the configure scripts:

\begin{verbatim}
$ cd unisim_simulators
$ ./build-configure.sh
\end{verbatim}

In case of failure while running \texttt{build-configure.sh}, check that \texttt{find}, \texttt{autoconf}, \texttt{autoheader} and \texttt{automake} are correctly installed.

\subsubsection{Configuring}

\begin{verbatim}
$ cd unisim_simulators
$ ./configure --prefix=${HOME}/unisim \
              --with-unisim-lib=${HOME}/unisim \
              --with-systemc=${HOME}/systemc \
              --with-tlm20=${HOME}/TLM-2008-06-09
\end{verbatim}

Script configure also supports the following options:
\begin{itemize}
\item \texttt{--with-sdl=<path-to-sdl-install>}: Override the search path for SDL C headers and libraries
\item \texttt{--with-boost=<path-to-boost-install>}: Override the search path for boost C++ headers and libraries
\item \texttt{--with-libxml2=<path-to-libxml2-install>}: Override the search path for libxml2 C headers and libraries
\item \texttt{--with-zlib=<path-to-zlib-install>}: Override the search path for zlib C headers and libraries
\item \texttt{--with-libedit=<path-to-libedit-install>}: Override the search path for libedit C headers and libraries
\item \texttt{--with-ncurses=<path-to-ncurses-install>}: Override the search path for ncurses C headers and libraries
\item \texttt{--with-systemc=<path-to-systemc-install>}: Override the search path for SystemC C++ headers and library. Depending on you host machine, script configure will search for library libsystemc.a in subdirectory lib-macosx, lib-gccsparcOS5, lib-cygwin, lib-linux, lib-linux64, lib-gcchpux11 (lib-macosx-x86 and lib-mingw32 if you patched your SystemC)
\item \texttt{--with-tlm20=<path-to-tlm2.0-install>}: Override the search path for TLM 2.0 C++ headers.
\item \texttt{--with-cacti=<path-to-cacti4.2-install>}: Override the search path for Cacti 4.2 C++ headers and libraries
\item \texttt{--enable-static}: build statically linked simulators if possible
\item \texttt{--enable-release}: Compile both debug and release versions of the simulators. Debug versions are more verbose than release versions
\item \texttt{--disable-<directory>}: Disable compilation of a whole subdirectory\\ (e.g. \texttt{--disable-simulator-tlm-ppcemu} prevents configure and make from entering directory tlm/ppcemu)
\end{itemize}

\subsubsection{Compiling the source code}

\begin{verbatim}
$ make
\end{verbatim}

\subsubsection{Installing the UNISIM Simulators}

\begin{verbatim}
$ make install
\end{verbatim}
