\subsection{Random Tests}
\label{sec:random_tests}

The validation of all the instructions is an impossible task as it would mean to check all the instructions with all the possible inputs.
So in order to create an extensive and systematic validation of all the instructions we have created a small tool to generate random tests.
The tool takes as input a template describing the instruction to test, but not defining the instruction inputs.
The tool then creates multiple tests of the instruction defining random inputs.
The outputs of the tests are printed on the screen, so we can compare the outputs of the simulator against those of the target platform.

\begin{figure}[!h]
	\begin{center}
\scriptsize
% \begin{tabular}{|p{0.8\textwidth}|}
\begin{tabular}{|l|}
\hline
\texttt{}\\
\texttt{ 1~eor~r9,~r9,~r9;~
}\\
\texttt{ 2~mov~r10,~\%u4;
}\\
\texttt{ 3~mov~r10,~r10,~ror~\#4;
}\\
\texttt{ 4~msr~cpsr\_f,~r10;
}\\
\texttt{ 5~adc\%(,eq,ne,cs,cc,mi,pl,vs,vc,hi,ls,ge,lt,gt,le,al)\%(,s)~\
}\\
\texttt{~~~~~~~~~~~~~~r9,~\%r,~\%s8,4,2;
}\\
\texttt{ 6~mrs~r10,~cpsr;
}\\
\texttt{ 7~mov~r10,~r10,~ror~\#28;
}\\
\texttt{ 8~and~\%R,~r10,~\#15;
}\\
\texttt{ 9~mov~\%R,~r9
}\\
\texttt{}\\
\hline
\end{tabular}
\end{center}

	\caption{\texttt{adc} with immediate input template.}
	\label{fig:adc_imm_template}
\end{figure}

Figure~\ref{fig:adc_imm_template} shows an example of template for the \texttt{adc} instruction using an immediate as input.
Line \texttt{1} sets to 0 the \texttt{r9} register which will be used later for storing the result of the \texttt{adc} instruction.
Lines \texttt{2}, \texttt{3} and \texttt{4} set randomly the status bits of the \texttt{cpsr} register which may influence the computation of the \texttt{adc} instruction.
Line \texttt{5} is the actual test of the \texttt{adc} instruction. 
The \texttt{\%(,eq,ne,...)} string appends nothing or one of the guarding pnemonics to the \texttt{adc} instruction, which one is selected randomly by the random tests generator.
Similarly the \texttt{\%(,s)} string appends nothing or the ``\texttt{s}'' string to the \texttt{adc} instruction.
The ``\texttt{s}'' flag indicates that the \texttt{cpsr} register should be updated if present, or leaved untouched otherwise.
The result of the \texttt{adc} instruction is stored in the \texttt{r9} register.
The \texttt{adc} uses two different inputs: a register and an immediate.
The register input is generated randomly by the test generator with the ``\texttt{\%r}'' string, choosing randomly one of the available registers and setting it to a random generated value.
The immediate input is generated randomly with the ``\texttt{\%s8,4,2}'' string\footnote{We are not going to get into details on how the immediate value is defined, just mention that it generates an integer that can be accepted by the ARM7 compiler.}.
Finally, lines \texttt{6}, \texttt{7} and \texttt{8} store the maybe modified \texttt{cpsr} status flags into a register chosen by the test generator (indicated with the ``\texttt{\%R}'').
Similarly, in line \texttt{9}, the register \texttt{r9} is copied into a register chosen by the test generator.
The registers marked with the ``\texttt{\%R}'' are then printed into the terminal, so we can compare the outputs from the simulation and the execution on ARM9 platform.
For each instruction test we check that the output register and the \texttt{cpsr} flags are correctly set.

\subsubsection{Results}

Test templates as the one presented in Figure~\ref{fig:adc_imm_template} have been developed for all the versions of the \texttt{adc} instruction (i.e., with two input registers, with a shifted immediate as input, and with a shifted register as input), and for all of the versions of most of the \textit{user level} instructions of the ARM7TDMI.
A total of 100,000 tests were generated from each template (making an average of 400,000 tests for instruction), and all the tests were succesfully passed. The following is a list of the validated instructions:
\begin{enumerate}
	\item adc
	\item add
	\item and
	\item bic
	\item cmn
	\item cmp
	\item eor
	\item mla
	\item mov
	\item mul
	\item mvn
	\item orr
	\item rsb
	\item rsc
	\item sbc
	\item smlal
	\item smull
	\item sub
	\item teq
	\item tst
	\item umlal
	\item umull
\end{enumerate}

This list includes all the \textit{user level} instructions of the ARM7TDMI processor, with the exception of memory instructions, branch instructions, and the system call instruction.
To validate those instructions (and complete the validation of the \textit{user level} instructions tested by the random tests), we performed tests over full applications presented in the following section.
