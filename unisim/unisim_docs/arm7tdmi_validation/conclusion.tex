\section{Conclusion}
\label{sec:conclusion}

This document has presented the different tests that have been performed in order to validate the ARM7TDMI simulator.
With the SPEC2000 application suit the simulator have been validated to be used with real world applications.
Thanks to the random tests technique developped at CEA an extensive validation of all the ARM7TDMI user level instructions have been done.
Finally, with the development of a small kernel the simulator has been validated for the execution of system level applications, like operating systems, drivers, and embedded applications.

As for any software application, the validation does not ensure that the simulator is bug free.
However, the validation performed ensures that the simulator functionality matches the ARM7TDMI functionality, and that any application can be run on it with confidence.
Additionally, the ARM7TDMI simulator provided in UNISIM (\url{http://www.unisim.org}) is open source, which means that users can fixe it if necessary and that corrections to the simulator can be easily shared.
